% Options for packages loaded elsewhere
\PassOptionsToPackage{unicode}{hyperref}
\PassOptionsToPackage{hyphens}{url}
%
\documentclass[
]{article}
\usepackage{amsmath,amssymb}
\usepackage{iftex}
\ifPDFTeX
  \usepackage[T1]{fontenc}
  \usepackage[utf8]{inputenc}
  \usepackage{textcomp} % provide euro and other symbols
\else % if luatex or xetex
  \usepackage{unicode-math} % this also loads fontspec
  \defaultfontfeatures{Scale=MatchLowercase}
  \defaultfontfeatures[\rmfamily]{Ligatures=TeX,Scale=1}
\fi
\usepackage{lmodern}
\ifPDFTeX\else
  % xetex/luatex font selection
\fi
% Use upquote if available, for straight quotes in verbatim environments
\IfFileExists{upquote.sty}{\usepackage{upquote}}{}
\IfFileExists{microtype.sty}{% use microtype if available
  \usepackage[]{microtype}
  \UseMicrotypeSet[protrusion]{basicmath} % disable protrusion for tt fonts
}{}
\makeatletter
\@ifundefined{KOMAClassName}{% if non-KOMA class
  \IfFileExists{parskip.sty}{%
    \usepackage{parskip}
  }{% else
    \setlength{\parindent}{0pt}
    \setlength{\parskip}{6pt plus 2pt minus 1pt}}
}{% if KOMA class
  \KOMAoptions{parskip=half}}
\makeatother
\usepackage{xcolor}
\usepackage[margin=1in]{geometry}
\usepackage{color}
\usepackage{fancyvrb}
\newcommand{\VerbBar}{|}
\newcommand{\VERB}{\Verb[commandchars=\\\{\}]}
\DefineVerbatimEnvironment{Highlighting}{Verbatim}{commandchars=\\\{\}}
% Add ',fontsize=\small' for more characters per line
\usepackage{framed}
\definecolor{shadecolor}{RGB}{248,248,248}
\newenvironment{Shaded}{\begin{snugshade}}{\end{snugshade}}
\newcommand{\AlertTok}[1]{\textcolor[rgb]{0.94,0.16,0.16}{#1}}
\newcommand{\AnnotationTok}[1]{\textcolor[rgb]{0.56,0.35,0.01}{\textbf{\textit{#1}}}}
\newcommand{\AttributeTok}[1]{\textcolor[rgb]{0.13,0.29,0.53}{#1}}
\newcommand{\BaseNTok}[1]{\textcolor[rgb]{0.00,0.00,0.81}{#1}}
\newcommand{\BuiltInTok}[1]{#1}
\newcommand{\CharTok}[1]{\textcolor[rgb]{0.31,0.60,0.02}{#1}}
\newcommand{\CommentTok}[1]{\textcolor[rgb]{0.56,0.35,0.01}{\textit{#1}}}
\newcommand{\CommentVarTok}[1]{\textcolor[rgb]{0.56,0.35,0.01}{\textbf{\textit{#1}}}}
\newcommand{\ConstantTok}[1]{\textcolor[rgb]{0.56,0.35,0.01}{#1}}
\newcommand{\ControlFlowTok}[1]{\textcolor[rgb]{0.13,0.29,0.53}{\textbf{#1}}}
\newcommand{\DataTypeTok}[1]{\textcolor[rgb]{0.13,0.29,0.53}{#1}}
\newcommand{\DecValTok}[1]{\textcolor[rgb]{0.00,0.00,0.81}{#1}}
\newcommand{\DocumentationTok}[1]{\textcolor[rgb]{0.56,0.35,0.01}{\textbf{\textit{#1}}}}
\newcommand{\ErrorTok}[1]{\textcolor[rgb]{0.64,0.00,0.00}{\textbf{#1}}}
\newcommand{\ExtensionTok}[1]{#1}
\newcommand{\FloatTok}[1]{\textcolor[rgb]{0.00,0.00,0.81}{#1}}
\newcommand{\FunctionTok}[1]{\textcolor[rgb]{0.13,0.29,0.53}{\textbf{#1}}}
\newcommand{\ImportTok}[1]{#1}
\newcommand{\InformationTok}[1]{\textcolor[rgb]{0.56,0.35,0.01}{\textbf{\textit{#1}}}}
\newcommand{\KeywordTok}[1]{\textcolor[rgb]{0.13,0.29,0.53}{\textbf{#1}}}
\newcommand{\NormalTok}[1]{#1}
\newcommand{\OperatorTok}[1]{\textcolor[rgb]{0.81,0.36,0.00}{\textbf{#1}}}
\newcommand{\OtherTok}[1]{\textcolor[rgb]{0.56,0.35,0.01}{#1}}
\newcommand{\PreprocessorTok}[1]{\textcolor[rgb]{0.56,0.35,0.01}{\textit{#1}}}
\newcommand{\RegionMarkerTok}[1]{#1}
\newcommand{\SpecialCharTok}[1]{\textcolor[rgb]{0.81,0.36,0.00}{\textbf{#1}}}
\newcommand{\SpecialStringTok}[1]{\textcolor[rgb]{0.31,0.60,0.02}{#1}}
\newcommand{\StringTok}[1]{\textcolor[rgb]{0.31,0.60,0.02}{#1}}
\newcommand{\VariableTok}[1]{\textcolor[rgb]{0.00,0.00,0.00}{#1}}
\newcommand{\VerbatimStringTok}[1]{\textcolor[rgb]{0.31,0.60,0.02}{#1}}
\newcommand{\WarningTok}[1]{\textcolor[rgb]{0.56,0.35,0.01}{\textbf{\textit{#1}}}}
\usepackage{graphicx}
\makeatletter
\def\maxwidth{\ifdim\Gin@nat@width>\linewidth\linewidth\else\Gin@nat@width\fi}
\def\maxheight{\ifdim\Gin@nat@height>\textheight\textheight\else\Gin@nat@height\fi}
\makeatother
% Scale images if necessary, so that they will not overflow the page
% margins by default, and it is still possible to overwrite the defaults
% using explicit options in \includegraphics[width, height, ...]{}
\setkeys{Gin}{width=\maxwidth,height=\maxheight,keepaspectratio}
% Set default figure placement to htbp
\makeatletter
\def\fps@figure{htbp}
\makeatother
\setlength{\emergencystretch}{3em} % prevent overfull lines
\providecommand{\tightlist}{%
  \setlength{\itemsep}{0pt}\setlength{\parskip}{0pt}}
\setcounter{secnumdepth}{-\maxdimen} % remove section numbering
\ifLuaTeX
  \usepackage{selnolig}  % disable illegal ligatures
\fi
\IfFileExists{bookmark.sty}{\usepackage{bookmark}}{\usepackage{hyperref}}
\IfFileExists{xurl.sty}{\usepackage{xurl}}{} % add URL line breaks if available
\urlstyle{same}
\hypersetup{
  pdftitle={Challenge-3},
  pdfauthor={Insert your name here},
  hidelinks,
  pdfcreator={LaTeX via pandoc}}

\title{Challenge-3}
\author{Insert your name here}
\date{2023-09-14}

\begin{document}
\maketitle

\hypertarget{i.-questions}{%
\subsection{I. Questions}\label{i.-questions}}

\hypertarget{question-1-emoji-expressions}{%
\paragraph{Question 1: Emoji
Expressions}\label{question-1-emoji-expressions}}

Imagine you're analyzing social media posts for sentiment analysis. If
you were to create a variable named ``postSentiment'' to store the
sentiment of a post using emojis (😄 for positive, 😐 for neutral, 😢
for negative), what data type would you assign to this variable? Why?
(\emph{narrative type question, no code required})

\textbf{Solution:} \emph{Delete this text and insert your answer here}

\hypertarget{question-2-hashtag-havoc}{%
\paragraph{Question 2: Hashtag Havoc}\label{question-2-hashtag-havoc}}

In a study on trending hashtags, you want to store the list of hashtags
associated with a post. What data type would you choose for the variable
``postHashtags''? How might this data type help you analyze and
categorize the hashtags later? (\emph{narrative type question, no code
required})

\textbf{Solution:} \emph{Delete this text and insert your answer here}

\hypertarget{question-3-time-travelers-log}{%
\paragraph{Question 3: Time Traveler's
Log}\label{question-3-time-travelers-log}}

You're examining the timing of user interactions on a website. Would you
use a numeric or non-numeric data type to represent the timestamp of
each interaction? Explain your choice (\emph{narrative type question, no
code required})

\textbf{Solution:} \emph{Delete this text and insert your answer here}

\hypertarget{question-4-event-elegance}{%
\paragraph{Question 4: Event Elegance}\label{question-4-event-elegance}}

You're managing an event database that includes the date and time of
each session. What data type(s) would you use to represent the session
date and time? (\emph{narrative type question, no code required})

\textbf{Solution:} \emph{Delete this text and insert your answer here}

\hypertarget{question-5-nominee-nominations}{%
\paragraph{Question 5: Nominee
Nominations}\label{question-5-nominee-nominations}}

You're analyzing nominations for an online award. Each participant can
nominate multiple candidates. What data type would be suitable for
storing the list of nominated candidates for each participant?
(\emph{narrative type question, no code required})

\textbf{Solution:} \emph{Delete this text and insert your answer here}

\hypertarget{question-6-communication-channels}{%
\paragraph{Question 6: Communication
Channels}\label{question-6-communication-channels}}

In a survey about preferred communication channels, respondents choose
from options like ``email,'' ``phone,'' or ``social media.'' What data
type would you assign to the variable ``preferredChannel''?
(\emph{narrative type question, no code required})

\textbf{Solution:} \emph{Delete this text and insert your answer here}

\hypertarget{question-7-colorful-commentary}{%
\paragraph{Question 7: Colorful
Commentary}\label{question-7-colorful-commentary}}

In a design feedback survey, participants are asked to describe their
feelings about a website using color names (e.g., ``warm red,'' ``cool
blue''). What data type would you choose for the variable
``feedbackColor''? (\emph{narrative type question, no code required})

\textbf{Solution:} \emph{Delete this text and insert your answer here}

\hypertarget{question-8-variable-exploration}{%
\paragraph{Question 8: Variable
Exploration}\label{question-8-variable-exploration}}

Imagine you're conducting a study on social media usage. Identify three
variables related to this study, and specify their data types in R.
Classify each variable as either numeric or non-numeric.

\textbf{Solution:} \emph{Delete this text and insert your answer here}

\hypertarget{question-9-vector-variety}{%
\paragraph{Question 9: Vector Variety}\label{question-9-vector-variety}}

Create a numeric vector named ``ages'' containing the ages of five
people: 25, 30, 22, 28, and 33. Print the vector.

\textbf{Solution:}

\begin{Shaded}
\begin{Highlighting}[]
\CommentTok{\# Enter code here}
\NormalTok{ages }\OtherTok{\textless{}{-}} \FunctionTok{c}\NormalTok{(}\DecValTok{25}\NormalTok{, }\DecValTok{30}\NormalTok{, }\DecValTok{22}\NormalTok{, }\DecValTok{28}\NormalTok{, }\DecValTok{33}\NormalTok{)}

\FunctionTok{print}\NormalTok{(ages)}
\end{Highlighting}
\end{Shaded}

\begin{verbatim}
## [1] 25 30 22 28 33
\end{verbatim}

\hypertarget{question-10-list-logic}{%
\paragraph{Question 10: List Logic}\label{question-10-list-logic}}

Construct a list named ``student\_info'' that contains the following
elements:

\begin{itemize}
\item
  A character vector of student names: ``Alice,'' ``Bob,'' ``Catherine''
\item
  A numeric vector of their respective scores: 85, 92, 78
\item
  A logical vector indicating if they passed the exam: TRUE, TRUE, FALSE
\end{itemize}

Print the list.

\textbf{Solution:}

\begin{Shaded}
\begin{Highlighting}[]
\CommentTok{\# Enter code here}
\NormalTok{student\_names }\OtherTok{\textless{}{-}} \FunctionTok{c}\NormalTok{(}\StringTok{"Alice"}\NormalTok{, }\StringTok{"Bob"}\NormalTok{, }\StringTok{"Catherine"}\NormalTok{)}

\NormalTok{scores }\OtherTok{\textless{}{-}} \FunctionTok{c}\NormalTok{(}\DecValTok{85}\NormalTok{, }\DecValTok{92}\NormalTok{, }\DecValTok{78}\NormalTok{)}

\NormalTok{pass }\OtherTok{\textless{}{-}} \FunctionTok{c}\NormalTok{(}\ConstantTok{TRUE}\NormalTok{, }\ConstantTok{TRUE}\NormalTok{, }\ConstantTok{FALSE}\NormalTok{)}

\NormalTok{student\_info }\OtherTok{\textless{}{-}} \FunctionTok{list}\NormalTok{(student\_names, scores, pass)}

\FunctionTok{print}\NormalTok{(student\_info)}
\end{Highlighting}
\end{Shaded}

\begin{verbatim}
## [[1]]
## [1] "Alice"     "Bob"       "Catherine"
## 
## [[2]]
## [1] 85 92 78
## 
## [[3]]
## [1]  TRUE  TRUE FALSE
\end{verbatim}

\hypertarget{question-11-type-tracking}{%
\paragraph{Question 11: Type Tracking}\label{question-11-type-tracking}}

You have a vector ``data'' containing the values 10, 15.5, ``20'', and
TRUE. Determine the data types of each element using the typeof()
function.

\textbf{Solution:}

\begin{Shaded}
\begin{Highlighting}[]
\CommentTok{\# Enter code here}
\NormalTok{data }\OtherTok{\textless{}{-}} \FunctionTok{c}\NormalTok{(}\DecValTok{10}\NormalTok{, }\FloatTok{10.5}\NormalTok{, }\FloatTok{15.5}\NormalTok{, }\StringTok{"20"}\NormalTok{, }\ConstantTok{TRUE}\NormalTok{)}

\FunctionTok{typeof}\NormalTok{(data[}\DecValTok{1}\NormalTok{])}
\end{Highlighting}
\end{Shaded}

\begin{verbatim}
## [1] "character"
\end{verbatim}

\begin{Shaded}
\begin{Highlighting}[]
\FunctionTok{typeof}\NormalTok{(data[}\DecValTok{2}\NormalTok{])}
\end{Highlighting}
\end{Shaded}

\begin{verbatim}
## [1] "character"
\end{verbatim}

\begin{Shaded}
\begin{Highlighting}[]
\FunctionTok{typeof}\NormalTok{(data[}\DecValTok{3}\NormalTok{])}
\end{Highlighting}
\end{Shaded}

\begin{verbatim}
## [1] "character"
\end{verbatim}

\begin{Shaded}
\begin{Highlighting}[]
\FunctionTok{typeof}\NormalTok{(data[}\DecValTok{4}\NormalTok{])}
\end{Highlighting}
\end{Shaded}

\begin{verbatim}
## [1] "character"
\end{verbatim}

\begin{Shaded}
\begin{Highlighting}[]
\FunctionTok{typeof}\NormalTok{(data[}\DecValTok{5}\NormalTok{])}
\end{Highlighting}
\end{Shaded}

\begin{verbatim}
## [1] "character"
\end{verbatim}

\hypertarget{question-12-coercion-chronicles}{%
\paragraph{Question 12: Coercion
Chronicles}\label{question-12-coercion-chronicles}}

You have a numeric vector ``prices'' with values 20.5, 15, and ``25''.
Use explicit coercion to convert the last element to a numeric data
type. Print the updated vector.

\textbf{Solution:}

\begin{Shaded}
\begin{Highlighting}[]
\CommentTok{\# Enter code here}
\NormalTok{prices }\OtherTok{\textless{}{-}} \FunctionTok{c}\NormalTok{(}\FloatTok{20.5}\NormalTok{, }\DecValTok{15}\NormalTok{, }\StringTok{"25"}\NormalTok{)}

\NormalTok{prices }\OtherTok{\textless{}{-}} \FunctionTok{as.numeric}\NormalTok{(prices[}\DecValTok{3}\NormalTok{])}

\FunctionTok{print}\NormalTok{(prices)}
\end{Highlighting}
\end{Shaded}

\begin{verbatim}
## [1] 25
\end{verbatim}

\hypertarget{question-13-implicit-intuition}{%
\paragraph{Question 13: Implicit
Intuition}\label{question-13-implicit-intuition}}

Combine the numeric vector c(5, 10, 15) with the character vector
c(``apple'', ``banana'', ``cherry''). What happens to the data types of
the combined vector? Explain the concept of implicit coercion.

\textbf{Solution:}

\begin{Shaded}
\begin{Highlighting}[]
\CommentTok{\# Enter code here}
\NormalTok{numeric\_vector }\OtherTok{\textless{}{-}} \FunctionTok{c}\NormalTok{(}\DecValTok{5}\NormalTok{, }\DecValTok{10}\NormalTok{, }\DecValTok{15}\NormalTok{)}
\FunctionTok{typeof}\NormalTok{(numeric\_vector)}
\end{Highlighting}
\end{Shaded}

\begin{verbatim}
## [1] "double"
\end{verbatim}

\begin{Shaded}
\begin{Highlighting}[]
\NormalTok{chr\_vector }\OtherTok{\textless{}{-}} \FunctionTok{c}\NormalTok{(}\StringTok{"apple"}\NormalTok{, }\StringTok{"banana"}\NormalTok{, }\StringTok{"cherry"}\NormalTok{)}
\FunctionTok{typeof}\NormalTok{(chr\_vector)}
\end{Highlighting}
\end{Shaded}

\begin{verbatim}
## [1] "character"
\end{verbatim}

\begin{Shaded}
\begin{Highlighting}[]
\NormalTok{combined\_vector }\OtherTok{\textless{}{-}} \FunctionTok{c}\NormalTok{(}\DecValTok{5}\NormalTok{, }\DecValTok{10}\NormalTok{, }\DecValTok{15}\NormalTok{, }\StringTok{"apple"}\NormalTok{, }\StringTok{"banana"}\NormalTok{, }\StringTok{"cherry"}\NormalTok{)}
\FunctionTok{typeof}\NormalTok{(combined\_vector)}
\end{Highlighting}
\end{Shaded}

\begin{verbatim}
## [1] "character"
\end{verbatim}

\begin{Shaded}
\begin{Highlighting}[]
\CommentTok{\#Data type became chr. Implicit coercion is when R auto{-}converts values from 1 data type to another to perform operations/comparisons}
\end{Highlighting}
\end{Shaded}

\hypertarget{question-14-coercion-challenges}{%
\paragraph{Question 14: Coercion
Challenges}\label{question-14-coercion-challenges}}

You have a vector ``numbers'' with values 7, 12.5, and ``15.7''.
Calculate the sum of these numbers. Will R automatically handle the data
type conversion? If not, how would you handle it?

\textbf{Solution:}

\begin{Shaded}
\begin{Highlighting}[]
\CommentTok{\# Enter code here}

\NormalTok{numbers }\OtherTok{\textless{}{-}} \FunctionTok{c}\NormalTok{(}\DecValTok{7}\NormalTok{, }\FloatTok{12.5}\NormalTok{, }\StringTok{"15.7"}\NormalTok{)}

\CommentTok{\# does not auto handle conversion. Use explicit coercion to change chr data type to numeric before calc sum.}

\NormalTok{numbers }\OtherTok{\textless{}{-}} \FunctionTok{as.numeric}\NormalTok{(numbers)}
\FunctionTok{sum}\NormalTok{(numbers)}
\end{Highlighting}
\end{Shaded}

\begin{verbatim}
## [1] 35.2
\end{verbatim}

\hypertarget{question-15-coercion-consequences}{%
\paragraph{Question 15: Coercion
Consequences}\label{question-15-coercion-consequences}}

Suppose you want to calculate the average of a vector ``grades'' with
values 85, 90.5, and ``75.2''. If you directly calculate the mean using
the mean() function, what result do you expect? How might you ensure
accurate calculation?

\textbf{Solution:}

\begin{Shaded}
\begin{Highlighting}[]
\CommentTok{\# Enter code here}
\NormalTok{grades }\OtherTok{\textless{}{-}} \FunctionTok{c}\NormalTok{(}\DecValTok{85}\NormalTok{, }\FloatTok{90.5}\NormalTok{, }\StringTok{"75.2"}\NormalTok{)}

\CommentTok{\# would expect an error msg. convert chr data type to numeric before calc mean}
\FunctionTok{mean}\NormalTok{(grades)}
\end{Highlighting}
\end{Shaded}

\begin{verbatim}
## Warning in mean.default(grades): argument is not numeric or logical: returning
## NA
\end{verbatim}

\begin{verbatim}
## [1] NA
\end{verbatim}

\begin{Shaded}
\begin{Highlighting}[]
\NormalTok{grades }\OtherTok{\textless{}{-}} \FunctionTok{as.numeric}\NormalTok{(grades)}
\FunctionTok{mean}\NormalTok{(grades)}
\end{Highlighting}
\end{Shaded}

\begin{verbatim}
## [1] 83.56667
\end{verbatim}

\hypertarget{question-16-data-diversity-in-lists}{%
\paragraph{Question 16: Data Diversity in
Lists}\label{question-16-data-diversity-in-lists}}

Create a list named ``mixed\_data'' with the following components:

\begin{itemize}
\item
  A numeric vector: 10, 20, 30
\item
  A character vector: ``red'', ``green'', ``blue''
\item
  A logical vector: TRUE, FALSE, TRUE
\end{itemize}

Calculate the mean of the numeric vector within the list.

\textbf{Solution:}

\begin{Shaded}
\begin{Highlighting}[]
\CommentTok{\# Enter code here}
\NormalTok{numeric\_vector }\OtherTok{\textless{}{-}} \FunctionTok{c}\NormalTok{(}\DecValTok{10}\NormalTok{, }\DecValTok{20}\NormalTok{, }\DecValTok{30}\NormalTok{)}

\NormalTok{chr\_vector }\OtherTok{\textless{}{-}} \FunctionTok{c}\NormalTok{(}\StringTok{"red"}\NormalTok{, }\StringTok{"green"}\NormalTok{, }\StringTok{"blue"}\NormalTok{)}

\NormalTok{logical\_vector }\OtherTok{\textless{}{-}} \FunctionTok{c}\NormalTok{(}\ConstantTok{TRUE}\NormalTok{, }\ConstantTok{FALSE}\NormalTok{, }\ConstantTok{TRUE}\NormalTok{)}

\NormalTok{mixed\_data }\OtherTok{\textless{}{-}} \FunctionTok{list}\NormalTok{(numeric\_vector, chr\_vector, logical\_vector)}

\FunctionTok{mean}\NormalTok{(mixed\_data[[}\DecValTok{1}\NormalTok{]])}
\end{Highlighting}
\end{Shaded}

\begin{verbatim}
## [1] 20
\end{verbatim}

\begin{Shaded}
\begin{Highlighting}[]
\CommentTok{\#note the double [[]] to access the elements of the first element of the list. [] accesses sublists, [[]] accesses the actual element.}
\end{Highlighting}
\end{Shaded}

\hypertarget{question-17-list-logic-follow-up}{%
\paragraph{Question 17: List Logic
Follow-up}\label{question-17-list-logic-follow-up}}

Using the ``student\_info'' list from Question 10, extract and print the
score of the student named ``Bob.''

\textbf{Solution:}

\begin{Shaded}
\begin{Highlighting}[]
\CommentTok{\# Enter code here}
\NormalTok{student\_names }\OtherTok{\textless{}{-}} \FunctionTok{c}\NormalTok{(}\StringTok{"Alice"}\NormalTok{, }\StringTok{"Bob"}\NormalTok{, }\StringTok{"Catherine"}\NormalTok{)}

\NormalTok{scores }\OtherTok{\textless{}{-}} \FunctionTok{c}\NormalTok{(}\DecValTok{85}\NormalTok{, }\DecValTok{92}\NormalTok{, }\DecValTok{78}\NormalTok{)}

\NormalTok{pass }\OtherTok{\textless{}{-}} \FunctionTok{c}\NormalTok{(}\ConstantTok{TRUE}\NormalTok{, }\ConstantTok{TRUE}\NormalTok{, }\ConstantTok{FALSE}\NormalTok{)}

\NormalTok{student\_info }\OtherTok{\textless{}{-}} \FunctionTok{list}\NormalTok{(student\_names, scores, pass)}

\FunctionTok{print}\NormalTok{(student\_info)}
\end{Highlighting}
\end{Shaded}

\begin{verbatim}
## [[1]]
## [1] "Alice"     "Bob"       "Catherine"
## 
## [[2]]
## [1] 85 92 78
## 
## [[3]]
## [1]  TRUE  TRUE FALSE
\end{verbatim}

\begin{Shaded}
\begin{Highlighting}[]
\CommentTok{\#student\_info$scores[2] gives NULL output bc the vectors of the lists aren\textquotesingle{}t named. Use DOUBLE INDEXING to access instead. OR see submitted challenge for alternative ans.}

\NormalTok{student\_info[[}\DecValTok{2}\NormalTok{]][}\DecValTok{2}\NormalTok{]}
\end{Highlighting}
\end{Shaded}

\begin{verbatim}
## [1] 92
\end{verbatim}

\hypertarget{question-18-dynamic-access}{%
\paragraph{Question 18: Dynamic
Access}\label{question-18-dynamic-access}}

Create a numeric vector values with random values. Write R code to
dynamically access and print the last element of the vector, regardless
of its length.

\textbf{Solution:}

\begin{Shaded}
\begin{Highlighting}[]
\CommentTok{\# Enter code here}
\NormalTok{numeric\_vector }\OtherTok{\textless{}{-}} \FunctionTok{c}\NormalTok{(}\DecValTok{1}\NormalTok{, }\DecValTok{2}\NormalTok{, }\DecValTok{3}\NormalTok{, }\DecValTok{4}\NormalTok{, }\DecValTok{5}\NormalTok{)}

\NormalTok{last\_element }\OtherTok{\textless{}{-}}\NormalTok{ numeric\_vector[}\FunctionTok{length}\NormalTok{(numeric\_vector)]}
\FunctionTok{print}\NormalTok{(last\_element)}
\end{Highlighting}
\end{Shaded}

\begin{verbatim}
## [1] 5
\end{verbatim}

\hypertarget{question-19-multiple-matches}{%
\paragraph{Question 19: Multiple
Matches}\label{question-19-multiple-matches}}

You have a character vector words \textless- c(``apple'', ``banana'',
``cherry'', ``apple''). Write R code to find and print the indices of
all occurrences of the word ``apple.''

\textbf{Solution:}

\begin{Shaded}
\begin{Highlighting}[]
\CommentTok{\# Enter code here}
\NormalTok{words }\OtherTok{\textless{}{-}} \FunctionTok{c}\NormalTok{(}\StringTok{"apple"}\NormalTok{, }\StringTok{"banana"}\NormalTok{, }\StringTok{"cherry"}\NormalTok{, }\StringTok{"apple"}\NormalTok{)}

\FunctionTok{print}\NormalTok{(}\FunctionTok{which}\NormalTok{(words }\SpecialCharTok{==} \StringTok{"apple"}\NormalTok{))}
\end{Highlighting}
\end{Shaded}

\begin{verbatim}
## [1] 1 4
\end{verbatim}

\begin{Shaded}
\begin{Highlighting}[]
\CommentTok{\# note which command {-}\textgreater{} prints indices}
\end{Highlighting}
\end{Shaded}

\hypertarget{question-20-conditional-capture}{%
\paragraph{Question 20: Conditional
Capture}\label{question-20-conditional-capture}}

Assume you have a vector ages containing the ages of individuals. Write
R code to extract and print the ages of individuals who are older than
30.

\textbf{Solution:}

\begin{Shaded}
\begin{Highlighting}[]
\CommentTok{\# Enter code here}
\NormalTok{ages }\OtherTok{\textless{}{-}} \FunctionTok{c}\NormalTok{(}\DecValTok{10}\NormalTok{, }\DecValTok{20}\NormalTok{, }\DecValTok{30}\NormalTok{, }\DecValTok{40}\NormalTok{, }\DecValTok{50}\NormalTok{)}

\FunctionTok{print}\NormalTok{(ages[}\FunctionTok{which}\NormalTok{(ages }\SpecialCharTok{\textgreater{}} \DecValTok{30}\NormalTok{)])}
\end{Highlighting}
\end{Shaded}

\begin{verbatim}
## [1] 40 50
\end{verbatim}

\begin{Shaded}
\begin{Highlighting}[]
\CommentTok{\#qn18 \& 20 wan to access the element itself, NOT the indices of the element {-}\textgreater{} access via: vectorName[code command]}

\CommentTok{\#qn19 only wants indices {-}\textgreater{} access via: which code command only (helps extract indices), no vectorName[] since this wld extract the element instead}
\end{Highlighting}
\end{Shaded}

\hypertarget{question-21-extract-every-nth}{%
\paragraph{Question 21: Extract Every
Nth}\label{question-21-extract-every-nth}}

Given a numeric vector sequence \textless- 1:20, write R code to extract
and print every third element of the vector.

\textbf{Solution:}

\begin{Shaded}
\begin{Highlighting}[]
\CommentTok{\# Enter code here}
\NormalTok{sequence }\OtherTok{\textless{}{-}} \DecValTok{1}\SpecialCharTok{:}\DecValTok{20}

\NormalTok{third\_element }\OtherTok{\textless{}{-}}\NormalTok{ sequence[}\FunctionTok{seq}\NormalTok{(}\AttributeTok{from=}\DecValTok{3}\NormalTok{, }\AttributeTok{to=}\DecValTok{20}\NormalTok{, }\AttributeTok{by=}\DecValTok{3}\NormalTok{)]}

\CommentTok{\#[] used to extract the elements themselves, not the indices of the elements.}

\FunctionTok{print}\NormalTok{(third\_element)}
\end{Highlighting}
\end{Shaded}

\begin{verbatim}
## [1]  3  6  9 12 15 18
\end{verbatim}

\begin{Shaded}
\begin{Highlighting}[]
\CommentTok{\#submitted challenge has the same ans, just wo from= \& to=}
\end{Highlighting}
\end{Shaded}

\hypertarget{question-22-range-retrieval}{%
\paragraph{Question 22: Range
Retrieval}\label{question-22-range-retrieval}}

Create a numeric vector numbers with values from 1 to 10. Write R code
to extract and print the values between the fourth and eighth elements.

\textbf{Solution:}

\begin{Shaded}
\begin{Highlighting}[]
\CommentTok{\# Enter code here}
\NormalTok{vector }\OtherTok{\textless{}{-}} \FunctionTok{c}\NormalTok{(}\DecValTok{1}\SpecialCharTok{:}\DecValTok{10}\NormalTok{)}

\FunctionTok{print}\NormalTok{(vector[}\DecValTok{4}\SpecialCharTok{:}\DecValTok{8}\NormalTok{])}
\end{Highlighting}
\end{Shaded}

\begin{verbatim}
## [1] 4 5 6 7 8
\end{verbatim}

\hypertarget{question-23-missing-matters}{%
\paragraph{Question 23: Missing
Matters}\label{question-23-missing-matters}}

Suppose you have a numeric vector data \textless- c(10, NA, 15, 20).
Write R code to check if the second element of the vector is missing
(NA).

\textbf{Solution:}

\begin{Shaded}
\begin{Highlighting}[]
\CommentTok{\# Enter code here}
\NormalTok{data }\OtherTok{\textless{}{-}} \FunctionTok{c}\NormalTok{(}\DecValTok{10}\NormalTok{, }\ConstantTok{NA}\NormalTok{, }\DecValTok{15}\NormalTok{, }\DecValTok{20}\NormalTok{)}

\NormalTok{missing }\OtherTok{\textless{}{-}} \FunctionTok{is.na}\NormalTok{(data[}\DecValTok{2}\NormalTok{])}
\FunctionTok{print}\NormalTok{(missing)}
\end{Highlighting}
\end{Shaded}

\begin{verbatim}
## [1] TRUE
\end{verbatim}

\begin{Shaded}
\begin{Highlighting}[]
\CommentTok{\#rmb to specify [2] since qn wants 2nd element specifically}
\end{Highlighting}
\end{Shaded}

\hypertarget{question-24-temperature-extremes}{%
\paragraph{Question 24: Temperature
Extremes}\label{question-24-temperature-extremes}}

Assume you have a numeric vector temperatures with daily temperatures.
Create a logical vector hot\_days that flags days with temperatures
above 90 degrees Fahrenheit. Print the total number of hot days.

\textbf{Solution:}

\begin{Shaded}
\begin{Highlighting}[]
\CommentTok{\# Enter code here}
\NormalTok{daily\_temp }\OtherTok{\textless{}{-}} \FunctionTok{c}\NormalTok{(}\DecValTok{70}\NormalTok{, }\DecValTok{80}\NormalTok{, }\DecValTok{90}\NormalTok{, }\DecValTok{100}\NormalTok{, }\DecValTok{110}\NormalTok{)}

\NormalTok{hot\_days }\OtherTok{\textless{}{-}}\NormalTok{ daily\_temp }\SpecialCharTok{\textgreater{}}\DecValTok{90}
\FunctionTok{print}\NormalTok{(}\FunctionTok{sum}\NormalTok{(hot\_days))}
\end{Highlighting}
\end{Shaded}

\begin{verbatim}
## [1] 2
\end{verbatim}

\begin{Shaded}
\begin{Highlighting}[]
\CommentTok{\#note SUM command}
\CommentTok{\#Since TRUE is treated as 1 and FALSE as 0 in numeric operations, this effectively counts the number of TRUE values in the vector.}
\end{Highlighting}
\end{Shaded}

\hypertarget{question-25-string-selection}{%
\paragraph{Question 25: String
Selection}\label{question-25-string-selection}}

Given a character vector fruits containing fruit names, create a logical
vector long\_names that identifies fruits with names longer than 6
characters. Print the long fruit names.

\textbf{Solution:}

\begin{Shaded}
\begin{Highlighting}[]
\CommentTok{\# Enter code here}
\NormalTok{fruits }\OtherTok{\textless{}{-}} \FunctionTok{c}\NormalTok{(}\StringTok{"banana"}\NormalTok{, }\StringTok{"pineapple"}\NormalTok{, }\StringTok{"strawberry"}\NormalTok{, }\StringTok{"blueberry"}\NormalTok{)}

\NormalTok{long\_names }\OtherTok{\textless{}{-}} \FunctionTok{nchar}\NormalTok{(fruits) }\SpecialCharTok{\textgreater{}} \DecValTok{6} \CommentTok{\# \textless{}{-} is a logical vector since it\textquotesingle{}s either true/false}

\FunctionTok{print}\NormalTok{(fruits[long\_names])}
\end{Highlighting}
\end{Shaded}

\begin{verbatim}
## [1] "pineapple"  "strawberry" "blueberry"
\end{verbatim}

\hypertarget{question-26-data-divisibility}{%
\paragraph{Question 26: Data
Divisibility}\label{question-26-data-divisibility}}

Given a numeric vector numbers, create a logical vector divisible\_by\_5
to indicate numbers that are divisible by 5. Print the numbers that
satisfy this condition.

\textbf{Solution:}

\begin{Shaded}
\begin{Highlighting}[]
\CommentTok{\# Enter code here}
\NormalTok{numbers }\OtherTok{\textless{}{-}} \FunctionTok{c}\NormalTok{(}\DecValTok{1}\SpecialCharTok{:}\DecValTok{20}\NormalTok{)}

\NormalTok{divisible\_by\_5 }\OtherTok{\textless{}{-}}\NormalTok{ numbers }\SpecialCharTok{\%\%} \DecValTok{5} \SpecialCharTok{==} \DecValTok{0}

\FunctionTok{print}\NormalTok{(numbers[divisible\_by\_5])}
\end{Highlighting}
\end{Shaded}

\begin{verbatim}
## [1]  5 10 15 20
\end{verbatim}

\begin{Shaded}
\begin{Highlighting}[]
\CommentTok{\#note \%\% command}
\end{Highlighting}
\end{Shaded}

\hypertarget{question-27-bigger-or-smaller}{%
\paragraph{Question 27: Bigger or
Smaller?}\label{question-27-bigger-or-smaller}}

You have two numeric vectors vector1 and vector2. Create a logical
vector comparison to indicate whether each element in vector1 is greater
than the corresponding element in vector2. Print the comparison results.

\textbf{Solution:}

\begin{Shaded}
\begin{Highlighting}[]
\CommentTok{\# Enter code here}
\NormalTok{vector1 }\OtherTok{\textless{}{-}} \FunctionTok{c}\NormalTok{(}\DecValTok{1}\NormalTok{, }\DecValTok{7}\NormalTok{, }\DecValTok{3}\NormalTok{, }\DecValTok{4}\NormalTok{, }\DecValTok{65}\NormalTok{, }\DecValTok{45}\NormalTok{, }\DecValTok{30}\NormalTok{)}
\NormalTok{vector2 }\OtherTok{\textless{}{-}} \FunctionTok{c}\NormalTok{(}\DecValTok{3}\NormalTok{, }\DecValTok{6}\NormalTok{, }\DecValTok{5}\NormalTok{, }\DecValTok{47}\NormalTok{, }\DecValTok{89}\NormalTok{, }\DecValTok{90}\NormalTok{, }\DecValTok{54}\NormalTok{)}

\NormalTok{comparison\_vector }\OtherTok{\textless{}{-}}\NormalTok{ vector1 }\SpecialCharTok{\textgreater{}}\NormalTok{ vector2}
\FunctionTok{print}\NormalTok{(comparison\_vector)}
\end{Highlighting}
\end{Shaded}

\begin{verbatim}
## [1] FALSE  TRUE FALSE FALSE FALSE FALSE FALSE
\end{verbatim}

\end{document}
